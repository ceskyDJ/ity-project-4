\documentclass[a4paper, 11pt, final]{article}

\usepackage{times}
\usepackage[T1]{fontenc}
\usepackage[utf8]{inputenc}
\usepackage[english,czech]{babel}
\usepackage[left=2cm, text={17cm, 24cm}, top=3cm]{geometry}

\usepackage[unicode, hidelinks]{hyperref}

\providecommand{\uv}[1]{\quotedblbase #1 \textquotedblleft}
% Source: BibTeX's original documentation by Oren Patashnik
% (http://ftp.cvut.cz/tex-archive/biblio/bibtex/base/btxdoc.tex)
\def\BibTeX{\textrm{B\kern-.05em\textsc{i\kern-.025em b}\kern-.08em T\kern-.1667em\lower.7ex\hbox{E}\kern-.125emX}}

\begin{document}

\begin{titlepage}
\begin{center}
    \Huge \textsc{Vysoké učení technické v~Brně}\\
    \huge \textsc{Fakulta informačních technologií}
    
    \vspace{\stretch{0.382}}
    
    \LARGE Typografie a publikování\,--\,4. projekt\\
    % BibTeX style source (edited): BibTeX's original documentation by Oren Patashnik
    % (http://ftp.cvut.cz/tex-archive/biblio/bibtex/base/btxdoc.tex)
    \Huge Používání nástroje B\kern-.05em{\huge I}\kern-.025em{\huge B}\kern-.08em\TeX{} pro citace v \LaTeX{}u
    
    \vspace{\stretch{0.618}}
\end{center}

\Large \noindent \today \hfill Michal Šmahel
\end{titlepage}

\section{Citace v \LaTeX{}u}
\label{sec:latex-citations}

Při psaní jakéhokoliv obsahu nás dříve či později zastihne potřeba citovat použitou literaturu, aby nebyla porušena
citační etika. Pokud se autor rozhodne svou práci vysázet v systému \LaTeX{}, má pro tyto účely k dispozici prostředí
\texttt{thebibliography} a příkaz \verb|\cite|.

V prostředí \texttt{thebibliography} se pomocí specifikují využité literární prameny a návěští, pomocí nichž se
následně bude možné na tyto záznamy odkazovat ve vlastním textu. Toto prostředí je koncipováno obdobně jako
\texttt{itemize} (prostředí pro neuspořádaný seznam) s tím rozdílem, že se jednotlivé položky uvozují příkazem
\verb|\bibitem{návěští}| (nikoliv \verb|\item| jako v případě \texttt{itemize}).

K odkazování poté slouží příkaz \verb|\cite{návěští}|, kde se jako povinný parametr uvede návěští uvedené v příkazu
\verb|\bibitem|. Praktické použití poté vypadá např. takto:

\begin{verbatim}
    % Vlastní text
    \dots jak uvádí Hooke v~\cite{hook} i~jeho německý kolega prof.~Kreuz
    (viz~\cite{kreuz, neumann}), je celá situace dosud nejasná.
    
    % Literatura
    \begin{thebibliography}{9}
    \bibitem{kreuz} Kreuz, K. H.
        Mechanické vlastnosti plastů. M\"{u}nchen: Spring Verlag, 1978.
    \bibitem{hook} Hooke, C. J.
        The Linear Pressure. N. Y.: Addison-Wesley Publishing Company,
        1987.
    \bibitem{neumann} Neumann, J.
        Tlaková reverze $\pi$ v kovových materiálech. Hutnické listy,
        55, 1990, s. 12-51.
    \end{thebibliography}
\end{verbatim}

Tato kapitola a uvedený příklad vychází z knihy \LaTeX{} pro začátečníky \cite{rybicka-latex}.

\section{BibTeX}

Z příkladu v kapitole \ref{sec:latex-citations} je patrné, že je ruční způsob vkládání citací v \LaTeX{}u
značně pracný a je velmi jednoduché udělat chybu a literaturu odcitovat chybně. Pro zpříjemnění práce je
možné použít například \BibTeX{}, který lze definovat např. takto: \uv{BibTEX je nástroj, který usnadňuje
sestavení soupisu bibliografických citací při sazbě dokumentů za pomoci systému LATEX.} \cite{pysny-plaincz}

\BibTeX{} zastoupí především tu část práce s citacemi, kdy je třeba správně vysázet jednotlivé citace
do~prostředí \texttt{thebibliography}. Místo přemýšlení nad tím, jak se mají poskládat jednotlivé údaje
o citovaném díle, stačí pouze tyto údaje systematicky vložit do konfiguračního souboru s příponou
\texttt{.bib} a další práci nechat na \BibTeX{}u. Místo prostředí \texttt{thebibliography} se využijí
příkazy \verb|\bibliographystyle| (nastavuje soubor \texttt{.bst}\,--\,viz podkapitolu
\ref{subsec:sty-files}) a \verb|\bibliography| (vygeneruje prostředí \texttt{thebibliography} podle
souboru \texttt{.bib}\,--\,viz podkapitolu \ref{subsec:bib-files}). \cite{pysny-plaincz, patashnik-bibtexing}

Příklad z kapitoly \ref{sec:latex-citations} poté bude vypadat např. takto (v souboru \texttt{.tex}):

\begin{verbatim}
    % Vlastní text
    \dots jak uvádí Hooke v~\cite{hook} i~jeho německý kolega prof.~Kreuz
    (viz~\cite{kreuz, neumann}), je celá situace dosud nejasná.
    
    % Literatura (přípony souborů je možné vynechat)
    \bibliographystyle{czplain.bst}
    \bibliography{sources-db.bib}
\end{verbatim}

Více pozitiv, ale také negativa nástroje \BibTeX{} je možné nalézt v článku od J\"{u}rgena Fenna, který
se touto problematikou zabývá podrobněji (zejména v kapitole 1) \cite{fenn-managing-citations}.

\subsection{Nastavení stylu zobrazených citací}
\label{subsec:sty-files}

Pro určení, jakým způsobem se budou sázet citace, které \BibTeX{} generuje, se používají soubory \texttt{.bst}.
Měnit způsob vysázení citací je třeba zejména proto, že se v praxi používají různé konvence. \BibTeX{} obsahuje
několik standardních stylů citací\,--\,\texttt{plain}, \texttt{unsrt}, \texttt{alpha} a \texttt{abbrv}. Tyto
styly není třeba stahovat a přikládat ke zdrojovým kódům \LaTeX{}u pro kompilaci. \cite{patashnik-bibtexing}

Kromě těchto předdefinovaných stylů je možné vytvořit si svůj vlastní\,--\,odvozený od výchozích stylů
uvedených výše. Tento proces je více rozepsán v dokumentaci \BibTeX{}u \cite{patashnik-bibtexing}. Postup
také podrobněji popisuje Ing.\,Pyšný ve své bakalářské práci, která se věnuje tvorbě stylu \texttt{czplain} 
\cite{pysny-plaincz}.

V českém prostředí je možné kromě stylu \texttt{czplain} \cite{pysny-plaincz} využít také styl \texttt{czechiso},
který vytvořil a zdokumentoval Ing.\,Martinek \cite{martinek-czechiso}. Další alternativou je styl
\texttt{csplainnat}\footnote{K tomuto stylu se neváže žádná textová práce, zdrojové kódy včetně příkladů
jsou dostupné na Githubu: \url{https://github.com/mudrd8mz/csplainnat}.}. Nejnovějším stylem vyvíjeným
pro aktuální normu ISO 690 platnou v České republice je \texttt{biblatex-iso690} \cite{hoftich-biblatex-iso690}.
Tento styl je však již pro \texttt{biblatex}, což je \LaTeX{}ový balíček, který je možné pokládat za nástupce
\BibTeX{}u. \cite{cstug-zpravodaj-2016-1}

\subsection{Soubor *.bib}
\label{subsec:bib-files}

Citovaná literatura je obsažena v souboru s příponou \texttt{.bib}. Ten pro každý zdroj obsahuje položky,
které musí být uvedeny v citaci, a z nichž následně \BibTeX{} vygeneruje položky \verb|\bibitem|. Položky
se různí podle toho, o jaký druh literatury jde. \BibTeX{} zvládá většinu běžných typů dokumentů, mezi něž
patří kvalifikační práce, odborné články, knihy, příspěvky z konferencí a další.
\cite{pysny-plaincz, patashnik-bibtexing}

Konkrétní příklady \texttt{.bib} souborů je možné nalézt například v bakalářské práci Ing.\,Pyšného\ 
\cite{pysny-plaincz}. Obecně položky popisuje dokumentace \BibTeX{}u \cite{patashnik-bibtexing}.

\subsection{Databáze citací}

Autoři, kteří píší větší množství publikací, na nichž případně spolupracují s dalšími lidmi, si nemusí
vystačit ani s \BibTeX{}em (případně balíčkem \texttt{biblatex}). V obou případech jsou totiž údaje
k citovaným zdrojům ukládány v souboru, který je lokálně umístěn na počítači autorů, v lepším případě
ve sdíleném repozitáři. Ani repozitář však neřeší problém s přenositelností často citované literatury
napříč několika dokumenty. \cite{sonke-opening-science, thummala-www-bib-tools}

Z tohoto důvodu se používají různé on-line služby, kde je možné vytvořit si vlastní databázi citované
literatury a případně ji i sdílet svým kolegům. To může velmi zefektivnit práci a usnadnit hledání
citací z často využívaných zdrojů. \cite{sonke-opening-science, amjad-jab-ref-ext, thummala-www-bib-tools}

\section{Alternativní řešení}

Byly zmíněny 2 alternativní cesty, kterými je možné se vydat, a to ruční řešení v \LaTeX{}u pomocí
vyplňování prostředí \texttt{thebibliography} položkami \verb|\bibitem| a použití možného nástupce
\BibTeX{}u\,--\,balíčku \texttt{biblatex}. Ruční řešení je vhodné využít pouze v případě menšího
počtu citovaných zdrojů. Jeho využití uplatní nejspíše začínající autoři, kteří nepotřebují tolik
zdrojů a zpočátku se nechtějí učit pracovat se složitějším systémem, jakým je např. \BibTeX{}.
Balíček \texttt{biblatex} vypadá jako zajímavá náhrada \BibTeX{}u. Problém může být s tím,
že definuje poněkud odlišné příkazy, takže není tak snadné na něho přejít z \BibTeX{}u.
Pro ty, kteří s \BibTeX{}em neumí by to však mohla být dobrá volba. Je však třeba zvážit
i další alternativy.

Mezi další alternativy patří např. jazyk CSL, balíček \texttt{opmac}. Více se o nich můžete
dozvědět ve Zpravodaji Československého sdružení uživatelů \TeX{}u \cite{cstug-zpravodaj-2016-1}.

Pro psaní dokumentů podle české normy nejlépe vychází využití \texttt{biblatex-iso690}, který
splňuje aktuální znění normy a je stále ve vývoji. \cite{cstug-zpravodaj-2016-1}

\section{Získávání údajů do citací}

I v dnešní době je občas stále problematické získat všechny potřebné údaje, které by měla citace
obsahovat. Ačkoliv existují snahy o vytvoření databází obsahujících seznam literatury
a k ní připravených citací v různých formátech, stále není možné jednoduše využít nějaký nástroj,
který by nalezl všechny nebo alespoň většinu písemných zdrojů. \cite{zakova-tracking-resources}

V současné době je mnoho informací obsaženo v databázích Google Scholar, Web of Science a Scopus.
Některé dokumenty jsou na více místech současně, avšak nezanedbatelný počet je pouze v jednom
z nich. Pro snazší vyhledávání vytvořila doc.\,Žáková aplikaci, kterou popisuje ve svém článku
pro konferenci 2016 Cybernetics \& Informatics. Tato aplikace by měla rovněž pomoct s hodnocením
autorů podle množství citací, které se váží k jejich publikacím. \cite{zakova-tracking-resources}

\newpage
\DeclareUrlCommand\url{\def\UrlLeft{<}\def\UrlRight{>} \urlstyle{tt}}
\bibliographystyle{my_czechiso.bst}
\bibliography{sources-db.bib}

\end{document}
